\chapter{Introducción}
Con la llegada, en 1906, del telescopio óptico de Galileo, se obtuvo el primer avance en las observaciones del universo, que en ese contexto se podían hacer únicamente empleando los ojos de manera directa. Se expandió por primera vez el universo observable para el hombre. Más adelante, nos dimos cuenta que el telescopio era un instrumento limitado para observar el universo, debido a que unicamente podiamos observar la región visible del espectro electromagnético \cite{Vazquez}. Más adelante se emplearían métodos novedosos para observar regiones, del espectro electromagnético, fuera del visible.  

El físico austriaco Víctor Hess entre los años 1911-1912 realizó experimentos cruciales, los cuales ponían en manifiesto la existencia de una radiación cuyo origen era del espacio exterior. Hess publicó los resultados de sus experimentos concluyendo lo siguiente: "Los resultados de estas observaciones parecen poder interpretarse admitiendo sencillamente que una radiación con gran poder de penetración procede de la parte superior de la atmósfera y, aunque progresivamente atenuada por ésta, produce, incluso en las zonas más bajas, una parte de la ionización observada en las cámaras cerradas. La intensidad de está radiación parece estar afectada por pequeñas variaciones aleatorias" \cite{Lugo}.

Hoy en día, se emplean distintos tipos de detectores para poder observar el universo en un rango más amplio de energía, todo esto con el fin de saber lo que hay y como ha ido evolucionando nuestro universo. Los rayos gamma de alta energía, por ejemplo, son producidos en fenómenos llamados GRB (destello de rayos gamma por sus siglas en inglés Gamma Ray Burst) que toman lugar en fenómenos muy violentos en universo como colisión de estrellas masivas, nucleos activos de galaxia, explosiones de estrellas tipo supernova, etc \cite{PEREZY2009}. Por lo que la detección de esta radiación proveniente del espacio exterior nos brinda aún más conocimiento sobre lo sucede en el universo y la evolución de este.